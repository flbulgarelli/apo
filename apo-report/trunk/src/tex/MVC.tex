\section{Contrucción de interfaces de usuario usando el patrón MVC}

\subsection{MVC}

Modelo Vista Controlador (MVC) es un patrón de arquitectura de software que
separa los datos de una aplicación, la interfaz de usuario, y la lógica de
negocio en tres componentes distintos.

El estilo fue descrito por primera vez en 1979 por \emph{Trygve Reenskaug},
entonces trabajando en Smalltalk en laboratorios de investigación de Xerox. 
La implementación original está escrita en Programación de Aplicaciones en 
Smalltalk-80(TM): Como utilizar Modelo Vista Controlador.

La idea principal de MVC, y que influyó a frameworks de presentación posteriores,
es la de Presentación Separada \emph{(Separated Presentation)} que consiste en
hacer unadivisión clara entre objetos de dominio que modelan nuestra percepción 
del mundo real y objetos de presentación que son los elementos Interfaz de
usuarios que vemos en lapantalla.\\

\includegraphics[width=300px, height=300px]{img/mvc}

\begin {itemize}


  	

\item {\bf Modelo}
	El modelo maneja el comportamiento y los datos del dominio de la aplicación,
	responde a los pedidos de informacion sobre su estado (por lo general de la
	vista), y responde a las instrucciones para cambiar su estado (por lo general
	desde el controlador). También deberían ser capaces de soportar	múltiples presentaciones
	
	
\item {\bf Vista}
	Muestra la informacion del modelo al usuario. 
	
\item {\bf Controlador}
	Es el intermediario entre el modelo y la vista.
	Mapea acciones del usuario con acciones al modelo.
		
	
\end {itemize}

\subsection{Eventos}

TODO\\

\subsection{Binding}
El Binding permite sincronizar los valores de las propiedades de dos objetos
diferentes (vista y modelo). Cada vez que el valor de una propiedad
cambia, el objeto notifica (lanza un evento), y todas las
propiedades que esten bindeadas a el reflejan los cambios automaticamente\\


\includegraphics[width=300px]{img/binding}

Asociando estas dos propiedades, el flujo entre ambos puede asociarce en dos
modos.

\begin {itemize}

\item {\bf OneWay}
Con este tipo de binding el flujo de datos se realiza en una sola dirección. 

\item {\bf TwoWay}
En este tipo de asociación el flujo se produce en ambas direcciones. Los cambios
realizados en el modelo se ven reflejados  en la vista y viceversa. (Este es el
tipo de binding que tenemos en el Arena)

\end {itemize}

\subsubsection{Ventajas}
Al tener vinvulados los datos entre la interfaz y el modelo

\subsubsection{DesVentajas}
Al no existir esa vinculacion, los datos que se tienen que pasar del modelo a la
interfaz y viceversa , hay que hacerlo a mano.


\subsubsection{Problemas sin binding}
Al no tener binding perdemos mucho tiempo de desarrollo en  traspasar los datos
de los objetos de dominio hacia los componentes de la interfaz grafica y
viceversa.

\subsubsection{Problemas con Binding}
Un problema que aparece frecuentemente es que el binding en su versión más 
sencilla modifica los objetos directamente, por lo que al cancelar una operación 
se debe volver al estado anterior, y este proceso es repetitivo y propenso a errores. 
Con frecuencia implica introducir comportamiento propio de la interfaz de
usuario, en mis objetos de dominio, mezclando la lógica de las dos partes de
la aplicación. Esto me obliga a adaptar mi dominio a este sistema, y repetir
muchas líneas de código.
	
\section{Transacciones}	

{ Muchas aplicacion trabajan con transacciones en la base de datos, sin embargo
el concepto excede al ambito de la bsae de datos y es un concepto util y
necesario para las interfaces de usuario.} 

Esta seccion intenta explicar la forma en que el concepto de transaccion
impacta en las interfaces de usuario\\

Las transacciones en un entorno de bases de datos tienen dos propositos
principales:

\begin {itemize}

  \item	
  Proveer unidades de trabajo confiables, que permitan mantener la consistencia
  incluso si el sistema falla.
  
  \item
  Proporcionar un aislamiento entre los programas de acceso a una base de datos
  al mismo tiempo.
  
\end{itemize}

¿Que pasa con las aplicaciones orientadas a objetos?
\begin {itemize}

  \item
  	Las unidades de trabajo y el aislamiento siguen siendo utiles en programacion
  	orientada a objetos.
  	
  \item
  	Pero la separacion entre el programa y la base de datos es contraria a los
  	principios de la programacion orientada a objetos.
\end{itemize}	

El objetivo de este trabajo es proponer una solución a estos dos problemas que
minimice el impacto en el código de las clases del dominio de la aplicación


