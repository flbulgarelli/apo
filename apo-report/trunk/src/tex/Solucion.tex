\section{Solución propuesta}
\label{sec:Solucion}

Utilizamos la Programación orientada a Aspectos para desarrollar dos
conceptos independientes, {\bf Observable} y  {\bf Transaccional}.

En las secciones subsiguientes se describe el comportamiento de los aspectos, y
en la sección \ref{sec:Union} la integración de ambos aspectos.

\subsection{Aspecto Observable}
	El aspecto Observable convierte los objetos de dominio en objetos observables,
	de una manera transparente y automática, es decir, disparan eventos cada vez
	que el valor de un atributo cambia.

\subsection{Aspecto Transaccional}
	El aspecto Transaccional convierte los objetos en objetos
	transaccionales, en forma transparente  y manteniendo su identidad.
	
	Toda operación sobre un objeto se realiza dentro de un contexto transaccional
	limitado por las operaciones detalladas en la sección
	\ref{ctxTransactional} (\emph{beginTransaction, commit, roolback}).
	Las operaciones realizadas a los objetos solo puede ser
	vistos dentro del mismo contexto y no modifican al objeto hasta que no se
	confirme la transacción (\emph{commit}).
	 
	El contexto transaccional permite tener un control en las operaciones de
	modificacion de los objetos, confirmar una operacion en curso, donde sus
	cambios seran impactados en el objeto, cancelar la todas las operaciones
	realizadas. Adamás soporta el trabajo concurrente, tambien llamado
	nivel de asislamieto, permitiendo a varios procesos realizar cambios en un
	mismo objeto al mismo tiempo. También soporta múltiples transacciones, ó transacciones
	anidadas.
	
	
\subsection{Unión de los Aspectos:}
\label{sec:Union}
La unión de los dos aspectos independientes es con el fin de asociar asociar los
eventos disparados por el dominio, con la translación actual.
 
