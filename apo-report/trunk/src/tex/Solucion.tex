\section{Solución propuesta}
\label{sec:Solucion}

	Utilizamos la programación orientada a aspectos para desarrollar dos
	conceptos independientes, {\bf Observable} y  {\bf Transaccional}.
	En las secciones subsiguientes se describe el comportamiento de cada uno de
	estos dos aspectos y en la sección \ref{sec:Union} la integración de ambos.

	\subsection{Aspecto Observable}
	\label{aspectoObservable}
		El aspecto Observable implementa un mecanismo transparente para que cuando un
		objeto de dominio cambia, cualquier parte del sistema pueda recibir una
		notificación informando ese cambio.
		Para ello, el sistema asocia cada atributo de un objeto que se desee observar
		con un conjunto de \emph{listeners}. 
		El aspecto intercepta todos los accesos a los atributos
		del objeto observado y, ante cada modificación de un atributo, notifica a todos
		los listeners que se han registrado para observar ese atributo.
		
		Este mecanismo permite asociar acciones específicas que se disparan cada vez
		que el objeto es modificado. En particular, se utiliza para mantener
		sincronizada la interfaz de usuario con los objetos del dominio.

	\subsection{Aspecto Transaccional}
	\label{aspectoTransaccional}
		El aspecto Transaccional permite controlar la visibilidad de las modificaciones
		realizadas a un objeto.
		Para ello, se define el concepto de \emph{contexto transaccional}.
		Un contexto trasaccional es un espacio de trabajo en el cual se pueden
		hacer modificaciones a un objeto que no serán vistas fuera de ese contexto.
	
		Los contextos transaccionales se delimitan por las
		operaciones detalladas en la Sección \ref{ctxTransactional}
		(\emph{beginTransaction}, \emph{commit} y \emph{rollback}).
		La operación de \emph{commit} confirma las modificaciones y las hace públicas
		fuera del contexto.
		Por su parte, la operación de \emph{rollback} provee un
		mecanismo automático para descartar todas las modificaciones realizadas
		dentro del contexto.
		Al ejecutarla, todos los objetos modificados dentro del contexto regresan al
		estado que tenían al comenzar la transacción.
		 
		El contexto transaccional permite el trabajo concurrente, permitiendo que
		varios procesos puedan acceder a un mismo objeto al mismo
		tiempo, con un aislamiento de nivel \emph{read commited}.
		 
		También se da soporte para \emph{transacciones anidadas}, es decir, la
		posibilidad de abrir un nuevo contexto transaccional dentro de otro contexto,
		denominado \emph{contexto padre}.
		Esta funcionalidad permite dividir una transacción en partes que pueden ser
		revertidas o confirmadas individualmente.
		
	\subsection{Integración de ambos aspectos entre sí y con la interfaz de usuario}
	\label{sec:Union}
		La unión de los dos aspectos independientes es con el fin de asociar asociar los
		eventos disparados por el dominio, con la translación actual.
		 
		Encontrar en la interfaz de usuario dónde se deben abrir y cerrar las
		transaccioens.
		Filtrar los eventos para que no le lleguen a una ventana los que no le
		corresponden por el contexto transaccional.
	
		 En Arena se integró los dos aspectos, el Observable y
		el transaccional, con el fin de que los objetos de dominio sean puros, y que no tengan la noción de
		eventos, ni transacciones, y así poder bindearlos con los componentes de de la
		interfaz gráfica. Y al cancelar la edición poder revertir los cambios
		transparéntenme.
	
		Como se explicó en la sección \ref{sec:Union}, para integrar los dos aspectos
		entre sí se requiere filtrar los eventos disparados por los objetos de dominio, 
		limitándolos a las ventanas que se encuentran dentro del mismo contexto
		transaccional.