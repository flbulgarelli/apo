\section{Solución propuesta}
\label{sec:Solucion}

Utilizamos la Programación orientada a Aspectos para desarrollar dos
conceptos independientes, {\bf Observable} y  {\bf Transaccional}.

En las secciones subsiguientes se describe el comportamiento de los aspectos, y
en la sección \ref{sec:Union} la integración de ambos aspectos.

\subsection{Aspecto Observable}
	El aspecto Observable permite que un objeto pueda notificar cada cambio de
	valor de sus atributos. A su vez, agrega el comportamiento de tener un conjunto
	de listener que estés escuchando esas notificaciones para poder reaccionar a su
	conveniencia.
	Dicha notificación se realiza utilizando eventos. En cada notificación el
	sistema le avisa a todos los listeners que se han registrado para observar ese
	atributo.

\subsection{Aspecto Transaccional}
	El aspecto Transaccional agrega  la funcionalidad de que un objeto pueda
	manejar los valores de sus variables de instancia, monitoreando sus cambios.	
	Toda operación sobre un objeto se realiza dentro de un \emph{contexto
	transaccional} limitado por las operaciones detalladas en la Sección
	\ref{ctxTransactional} (\emph{beginTransaction, commit, rollback}).
	Las operaciones realizadas a los objetos solo puede ser
	vistos dentro del mismo contexto y no modifican al objeto hasta que no se
	confirme la transacción (\emph{commit}).
	 
	El contexto transaccional permite tener un control en las operaciones de
	modificación de los objetos, confirmar una operación en curso, donde sus
	cambios serán impactados en el objeto, cancelar la todas las operaciones
	realizadas. Además soporta el trabajo concurrente, también llamado
	nivel de aislamiento, permitiendo a varios procesos realizar cambios en un
	mismo objeto al mismo tiempo. También soporta transacciones
	anidadas.
	
	
\subsection{Unión de los Aspectos:}
\label{sec:Union}
La unión de los dos aspectos independientes es con el fin de asociar asociar los
eventos disparados por el dominio, con la translación actual.
 
