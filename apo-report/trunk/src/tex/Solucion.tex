\section{Solución propuesta}
\label{Solucion}

	Basándonos en el paradigma AOP modelamos como aspectos dos conceptos: {\bf Observable} y  {\bf Transaccional}.
	En las secciones subsiguientes se describe el comportamiento de cada uno de
	estos dos aspectos y en la sección \ref{sec:Union} la integración de ambos.

	\subsection{Aspecto Observable}
	\label{aspectoObservable}
		El aspecto Observable implementa un mecanismo transparente para que cuando un
		objeto de dominio cambia, cualquier parte del sistema pueda recibir una
		notificación informando ese cambio.
		Para ello, el sistema permite asociar a cada propiedad de un objeto un conjunto de \emph{listeners}. 
		El aspecto intercepta modificación a una propiedad
		de un objeto observado y notifica a los listeners que se han registrado para observar esa propiedad.
		
		Este mecanismo permite asociar acciones específicas que se disparan cada vez
		que el objeto es modificado. En particular, se utiliza para mantener
		sincronizada la UI con los objetos del dominio.

	\subsection{Aspecto Transaccional}
	\label{aspectoTransaccional}
		El aspecto transaccional permite controlar la visibilidad de las modificaciones
		realizadas a un objeto.
		Llamamos \emph{objeto transaccional} a los objetos a los que se les aplica este aspecto.
		
		Para controlar la visibilidad de las modificaciones realizadas sobre los objetos transaccionales, 
		se define el concepto de \emph{contexto transaccional}.
		Un contexto transaccional es un espacio de trabajo en el cual se pueden
		realizar operaciones que no serán visibles fuera de ese contexto.
		Toda acción realizada dentro del sistema es asociada con algún contexto transaccional.
		
		El aspecto transaccional intercepta \textbf{en forma transparente} todos los accesos al
		estado interno de los objetos transaccionales, permitiendo la intervención del contexto.
		Cuando se modifica el estado interno de un objeto transaccional el nuevo valor es almacenado en el contexto y 
		el objeto permanece inalterado.
		Cuando se desea leer el estado interno de un objeto transaccional se toma el valor del contexto, en caso de existir.
		De esta manera las operaciones realizadas dentro del mismo contexto ven un estado del objeto que es distinto 
		al que ven las operaciones realizadas en cualquier otro contexto.
	
		Los contextos transaccionales se delimitan por las
		operaciones detalladas en la Sección \ref{ctxTransactional}
		(\emph{beginTransaction}, \emph{commit} y \emph{rollback}).
		La operación de \emph{commit} confirma las modificaciones y las hace públicas
		fuera del contexto.
		Por su parte, la operación de \emph{rollback} provee un
		mecanismo automático para descartar todas las modificaciones realizadas
		dentro del contexto.
		Al ejecutarla, todos los objetos modificados dentro del contexto regresan al
		estado que tenían al comenzar la transacción.
		 
		El contexto transaccional permite el trabajo concurrente, de esta forma
		varias operaciones dentro de un mismo sistema pueden acceder a un objeto transaccional al mismo
		tiempo con un aislamiento de nivel \emph{read commited}.
				 
		También se da soporte para \emph{transacciones anidadas}, es decir, la
		posibilidad de abrir un nuevo contexto transaccional dentro de otro contexto,
		denominado \emph{contexto padre}.
		Esta funcionalidad permite dividir una transacción en partes que pueden ser
		revertidas o confirmadas individualmente.
		
		La Sección \ref{TransactionalModel} se muestra detalladamente como trabaja el
		aspecto transaccional.
		
		
		
	\subsection{Integración de ambos aspectos entre sí y con la UI}
	\label{sec:Union}
		Al utilizar los aspectos Observable y Transaccional simultáneamente aparecen
		situaciones complejas que deben ser tenidas en cuenta. 
		También es necesario estudiar la forma en que puedan integrar ambos aspectos
		con la UI.
		Para atacar estos problemas se tomaron tres acciones:
		
		\begin{enumerate}
		  \item Cuando la UI muestra en pantalla el valor de una
		  propiedad de un objeto de dominio, debe registrarse como listener de esa
		  propiedad.
		  De esa forma, en caso de producirse un cambio en el valor de la propiedad, la
		  UI podrá reflejar esa modificación en forma inmediata.
		  
		  \item Siempre que desde la UI el usuario tenga la posibilidad de cancelar
		  una acción que comenzó a realizar, se debe utilizar un contexto
		  transaccional.
		  Esto se logra incluyendo en la UI las operaciones de
		  \emph{beginTransaction}, \emph{commit} y \emph{rollback}.
		  
		  Consideramos que la decisión de cuándo comenzar o terminar una transacción 
		  no puede ser decidida por nuestro framework de transacciones. 
		  La estrategia utilizada para integrar un framework
		  específico de UI con el aspecto transaccional es extender el framework
		  elegido con herramientas que manejen los contextos transaccionales en forma
		  automática.
		  
		  \item La integración de los aspectos entre sí se logra asociando a
		  cada listener con un contexto transaccional y limitando el alcance de los
		  eventos producidos por el aspecto observable para que sólo notifiquen a los
		  listeners que se encuentran en la mismo contexto transaccional que produjo el cambio.
		\end{enumerate}
