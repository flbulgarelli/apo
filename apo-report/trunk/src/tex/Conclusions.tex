\section{Conclusiones}
\label{conclusions}
A pesar de que hoy en día existe gran cantidad de herramientas para el
desarrollo de interfaces de usuario, existen problemas que
aparecen con frecuencia en los desarrollos industriales que no son resueltos por
dichas herramientas.
 Cuando esto sucede, los programadores deben recurrir a soluciones \emph{ad-hoc}
que con frecuencia implican atacar problemas generales de forma manual y
escribiendo grandes cantidades de código rutinario, que con frecuencia resulta
propenso a errores.

En este trabajo atacamos dos problemas rutinarios de las interfaces de usuarios:
por un lado la sincronización de datos entre la vista y el modelo de dominio; y
por otro la posibilidad de modelar operaciones que se realicen atómicamente.

%lo pudimos hacer y quedó algo piola
Utilizando las ideas de la programación orientada a aspectos pudimos desarrollar
una herramienta que soluciona ambos problemas en forma transparente, dado que no
requiere que hagamos modificaciones en el código del dominio, y genérica, dado
que es aplicable a un gran conjunto de posibles dominios.

%por qué es piola?
% qué cosas son más simples?

%pruebas, cómo nos convencemos de que lo que hicimos sirve
Se probaron las transacciones en un a aplicación mas grande que la del ejemplo
que se desarrollo en este trabajo. La aplicación que se utilizo el aspecto
transaccional fue en la de un Kiosco, escrita en su totalidad en scala  y
utilizando el framework Swing de Java para la interfaz de usuario.

% otras cosas útiles
Se desarrollo un sistema de monitorieo para las transacciones. Este sistema
permite debuggear las transacciones, proveyendo una interfaz que muestra los
objetos afectados por la transaccion.

Apo

%utilidad en la docencia.
Las mejoras que se realizaron en el Arena, simplifica el aprendizaje porque
antes los alumnos tenian que tirar los eventos a mano, y eso implicaba explicar
conceptos complicados y no permitia folalizarse en las cosas que se creen importantes.
Ademas el Arena a va se usado no solo en la Universiad Nacional de Quilmes,
sinó, En la Universidad Tecnológica Nacional (UTN) y en la Universidad Nacional
De San Martin (UNSAM)
