\section{Conclusiones}
\label{sec:conclusions}
A pesar de que hoy en día existe gran cantidad de herramientas para el
desarrollo de interfaces de usuario, existen problemas que
aparecen con frecuencia en los desarrollos industriales que no son resueltos por
dichas herramientas.
 Cuando esto sucede, los programadores deben recurrir a soluciones \emph{ad-hoc}
que con frecuencia implican atacar problemas generales de forma manual y
escribiendo grandes cantidades de código rutinario, que con frecuencia resulta
propenso a errores.

En este trabajo atacamos dos problemas rutinarios de las interfaces de usuarios:
por un lado la sincronización de datos entre la vista y el modelo de dominio; y
por otro la posibilidad de modelar operaciones que se realicen atómicamente.

%lo pudimos hacer y quedó algo piola
Utilizando las ideas de la programación orientada a aspectos pudimos desarrollar
una herramienta que soluciona ambos problemas en forma transparente, dado que no
requiere que hagamos modificaciones en el código del dominio, y genérica, dado
que es aplicable a un gran conjunto de posibles dominios.

%por qué es piola?
% qué cosas son más simples?
De cara a estos objetivos surgieron las siguientes cuestiones para destacar

\begin{quote}

	\begin{itemize}
	  
		\item Utilizar la programación orientada a aspectos para la resolución de
		estos problemas fue muy útil.
		
		\item  \ldots
		
		\item \ldots
		\item \ldots
	  
	\end{itemize}
	
\end{quote}

%pruebas, cómo nos convencemos de que lo que hicimos sirve
Se probaron las transacciones en una aplicación mas compleja como la de un
Kiosco.
La aplicación se realizo toda en scala-swing. Y donde mas se veía la utilidad de
las transacciones era en las ventas. Cuando estaba en una venta, se agregaban
productos, pero al momento de confirmar la venta, el cliente puede cancelar la
venta también, y en ese momento es donde le damos rollback y nada mas! \ldots
\comment{no se si va esto}

% otras cosas útiles
Y el tercer objetivo era
proveer una sistema de monitoreo para las transacciones.
¿Qué es lo bueno de esto?

Apo

extensiones al arena.

%utilidad en la docencia.
Mejoramos el Arena, lo van a usar en UNQ, UTN y UNSAM.
Simplifica el aprendizaje porque antes tenían que ver el tema eventos a mano, era
complicado. Permite focalizarse en otras cosas.
