\section{Conclusiones}
\label{sec:conclusions}

El presente trabajo tenia tres objetivos principales. El primero era automatizar
la comunicación entre el modelo de dominio y la interfaz de usuario. El segundo
objetivo era proveer un soporte para transacciones; Y el tercer objetivo era
proveer una sistema de monitoreo para las transacciones.

De cara a estos objetivos surgieron las siguientes cuestiones para destacar

\begin{quote}

	\begin{itemize}
	  
		\item Utilizar la programación orientada a aspectos para la resolución de
		estos problemas fue muy útil.
		
		\item  \ldots
		
		\item \ldots
		\item \ldots
	  
	\end{itemize}
	
\end{quote}

Se probo las transacciones en una aplicacion mas compleja como la de un Kiosco.
La aplicación se realizo toda en scala-swing. Y donde mas se veia la utilidad de
las transacciones era en las ventas. Cuando estaba en una venta, se agregaban
productos, pero al momento de confirmar la venta, el cliente puede cancelar la
venta tambien, y en ese momento es donde le damos rollback y nada mas! \ldots
\comment{no se si va esto}
