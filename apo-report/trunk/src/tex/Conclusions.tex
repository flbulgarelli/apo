\section{Conclusiones}
\label{conclusions}
Hoy en día existe gran cantidad de herramientas para el
desarrollo de interfaces de usuario. Sin embargo algunos problemas que
aparecen con frecuencia en los desarrollos industriales no son resueltos
adecuadamente por dichas herramientas.
 Cuando esto sucede, los programadores deben recurrir a soluciones
\emph{ad-hoc},  que con frecuencia implican atacar problemas generales de forma
manual y escribiendo grandes cantidades de código rutinario, que con
frecuencia resulta propenso a errores.

En este trabajo atacamos dos problemas rutinarios de las interfaces de usuarios:
por un lado la sincronización de datos entre la vista y el modelo de dominio; y
por otro la posibilidad de modelar operaciones que se realicen atómicamente.

%lo pudimos hacer y quedó algo piola
Utilizando las ideas de la programación orientada a aspectos pudimos desarrollar
una herramienta que soluciona ambos problemas en forma \emph{transparente}, dado
que no requiere que hagamos modificaciones en el código del dominio, y \emph{genérica},
dado que es aplicable a un gran conjunto de posibles dominios.

%pruebas, cómo nos convencemos de que lo que hicimos sirve
\medskip

Para poner a prueba el aspecto transaccional se lo utilizó en una aplicación de
mayor envergadura a los ejemplos mostrados en este trabajo.
Para eso se desarrolló la aplicación de un Kiosco, escrita en su totalidad en
scala y utilizando el framework Swing de Java para la interfaz de usuario.
Este desarrollo además nos permite comprobar que los aspectos están desacoplados
entre sí y pueden ser utilizados independientemente del framework Arena.

% otras cosas útiles
%Apo

Por otro lado, se desarrolló un sistema de monitoreo para las transacciones.
Este sistema permite debuggear las transacciones, proveyendo una interfaz que muestra los
objetos afectados por la transacción.


%utilidad en la docencia.
Las mejoras que se realizaron en el Arena simplifican el aprendizaje.
En las versiones previas los estudiantes tenían que tirar los eventos en forma
manual, y eso implicaba explicar conceptos complejos, distrayendo la atención de
los conceptos fundamentales de la materia.
A partir de estas modificaciones, el framework Arena será utilizado no
sólo en la Universidad Nacional de Quilmes, sino también en la Universidad
Tecnológica Nacional (UTN) y en la Universidad Nacional De San Martín (UNSAM).
