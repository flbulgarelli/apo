\section{Conclusiones}
\label{Conclusions}
Hoy en día existe gran cantidad de herramientas para el
desarrollo de UI. Sin embargo algunos problemas que son muy habituales en los desarrollos
industriales no son resueltos adecuadamente por dichas herramientas.
Cuando esto sucede, los programadores deben recurrir a soluciones
\emph{ad-hoc},  que muchas veces implican atacar problemas generales de forma
manual y escribiendo grandes cantidades de código rutinario, que con
frecuencia resulta propenso a errores.

En este trabajo abordamos dos problemas rutinarios de las UIs:
por un lado la sincronización de datos entre la vista y el modelo de dominio; y
por otro la posibilidad de modelar operaciones que se realicen atómicamente.

%lo pudimos hacer y quedó algo piola
Utilizando las ideas de la programación orientada a aspectos pudimos desarrollar
una herramienta que soluciona ambos problemas en forma \emph{transparente}, dado
que no requiere que hagamos modificaciones en el código del dominio, y \emph{genérica},
dado que es aplicable a un gran conjunto de posibles dominios.

%pruebas, cómo nos convencemos de que lo que hicimos sirve
\medskip

Para poner a prueba el aspecto transaccional, se lo aplicó en una aplicación de
mayor tamaño que los ejemplos mostrados en este trabajo.
Para eso se desarrolló la aplicación de un kiosco, escrita en su totalidad en
Scala y utilizando el framework Swing de Java para la UI.
Este desarrollo además nos permite comprobar que los aspectos están desacoplados
entre sí y pueden ser utilizados independientemente del framework Arena.

% otras cosas útiles
%Apo

Por otro lado, para dar soporte al desarrollo aprovechando el aspecto transaccional, 
se desarrolló un sistema de monitoreo para las transacciones.
Este sistema permite visualizar las transacciones, proveyendo una interfaz que muestra los
objetos afectados por la transacción.

\medskip

%utilidad en la docencia.
Finalmente, los aspectos transaccional y observable fueron integrados al framework Arena, 
favoreciendo su uso para la enseñanza de construcción de interfaces de usuario.
En las versiones previas los estudiantes tenían que disparar los eventos en forma
manual, y eso implicaba explicar conceptos complejos, distrayendo la atención de
los temas fundamentales de la materia.

Como resultado adicional de este trabajo, se extendió el framework agregando nuevos componentes y ejemplos. 
A partir de estas modificaciones, el framework Arena está siendo utilizado no
sólo en la Universidad Nacional de Quilmes, sino también en la Universidad
Tecnológica Nacional (UTN) y en la Universidad Nacional de San Martín (UNSAM).
