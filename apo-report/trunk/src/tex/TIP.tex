\section{Aplicación de ejemplo}
(Fijarse si se puede poner en la problematica)
1)Edicion con cancelar
2)multiple cancelaciones
3)buscador
4)monitor de transacciones


\section{Nuestra herramienta: TIP}
\label{sec:Jeronimo}



Programacion de los aspectos 
	Se utiliza javassist porque es independiente al usuario, en cambio si utilamos
	aspectJ obligamos a cambiar de entorno al usuario
	Abstraccion de javassist, y es independiente para poder utilizarlo
	
	El aspecto Transaccional esta basado en el que habia hecho nico y javi (contar
	que mejoras se hicieron)
	
	El aspecto Observable (contar que le agrega metodos y field)


Integracion con el domminio
	Facil de configurar con annotations
	Poder tener una u otro aspecto 


Integración de aspectos con el Arena
	Asocio un transaccion con una ventana.
	El trabajo de los eventos va al dominio y no al arena!
	Modificar el arena para que escuche los eventos que estan solo en su
	transaccion. (contar del transaccional dialog)
	

(Mejoras al arena | utilacion de scala => fijarse donde va)
	
	
(poder hacer otra implementacion de (eventos | Swing) y utilizarlo con las
transacciones (trabajo futuro si no se llega))

Contar que esta publicado, con la licencia y blah

Test de los aspectos




