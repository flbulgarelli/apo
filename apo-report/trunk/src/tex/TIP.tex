\section{Nuestra herramienta: }
\label{Implementacion}

	En esta sección se explica como se llevó a cabo la implementación de la solución
	propuesta en la Sección \ref{Solucion}, cumpliendo a su vez los objetivos
	planteados en la Sección \ref{Objective}. Para ello se desarrollaron dos
	herramientas:
	\emph{Pure Objects Observable} (POO) para atacar a la problemática de la
	observabilidad, y \emph{Pure Object Transaction} (POT) para atacar a la
	problemática transaccional.

	Todas las herramientas desarrolladas se encuentran publicadas bajo la licencia \emph{Creative Commons}.
	En la sección \ref{instalation} se encuentran las instrucciones para instalar la herramienta y 
	links a la documentación disponible, junto con algunos ejemplos de uso.
	
\subsection{Selección de un framework de aspectos}
	
\subsection{Selección de un framework de aspectos}  
		Un primer paso para la implementación de la herramienta fue la selección de una
		tecnología que permitiera desarrollar utilizando programación orientada a
		aspectos.
		Con ese objetivo, se evaluaron dos frameworks: Javassist
		\cite{chiba00loadtime} y AspectJ \cite{KiczalesHHKPG01}.
	
	\medskip 
	Encontramos que AspectJ es una herramienta de más alto nivel, que extiende
	el lenguaje Java agregando construcciones específicas para trabajar con
	los conceptos de la teoría de programación orientada a aspectos.
	Sin embargo, AspectJ requiere que el programador que use nuestro framework
	utilice un compilador específico, provisto por AspectJ. 
	Consideramos que esta característica es negativa, por condicionar el
	entorno de trabajo de los usuarios de nuestra herramienta.
	
	Javassist, por su lado, aplica los aspectos al momento de
	la carga de las clases, sólo requiere que utilicemos un \emph{ClassLoader}
	específico, resultando menos invasivo para los programadores de aplicaciones
	basadas en nuestra herramienta.
	Como aspecto negativo, notamos que es un framework de muy bajo nivel que
	carece de las abstracciones necesarias para modelar con aspectos y en cambio
	obliga a pensar a nivel de edición de expresiones en el bytecode de una clase
	compilada.
	
	Elegimos Javassist porque priorizamos minimizar el impacto hacia los usuarios
	de nuestra herramienta.
	Para minimizar los problemas asociados a utilizar un framework de tan bajo
	nivel, desarrollamos una herramienta que simplifica su uso agregando algunas
	abstracciones útiles. Esta herramienta se describe en la sección siguiente.

\subsection{Desarrollo de Aspect for Pure Objects}
	El framework Javassist permite modificar directamente el \emph{bytecode} de
	una clase en el momento de cargarla.
	Por ser de tan bajo nivel es uno de los frameworks de aspecto más poderosos,
	pero a su vez el código que requiere es muy poco entendible.
	Por eso se desarrolló una herramienta llamada \emph{Aspect for Pure Objects} (APO), 
	que permite definir aspectos utilizando conceptos de más alto nivel y
	aplicárselo a un grupo de objetos. 
	
	La Figura \ref{aopImage} muestra esquemáticamente el diseño de la herramienta.
	Una instancia de \code{AdviceWeaver} se ocupa de aplicar los cambios sobre las
	clases.
	Cada uno de los cambios que debe realizar el \code{AdviceWeaver} está
	modelado por un \code{Advice}, que consiste de un \code{PointCut} y un
	\code{JoinPoint}.
	
	El \code{PointCut} tiene la responsabilidad de determinar el conjunto de
	clases sobre el que aplica el advice. Se provee algunas implementaciones básicas
	de \code{PointCuts}, por ejemplo, \code{ClassPointCut}, \code{FieldPointCut}, 
	\code{MethodPointCut}, entre otros. Cada uno de estos \code{PointCuts} tienen un conjunto de funciones
	de filtro, programadas por el usuario programador que use herramienta. 

	El \code{JoinPoint} será el responsable de realizar las modificaciones 
	sobre las clases seleccionadas. Para ello, tiene una coleccion de \code{Interceptors}.
 	Los  \code{Interceptors} son objetos que interceptan una acción específica del código,  por ejemplo
 	llamadas a métodos, acceso a los atributos, ect. La herramienta provee algunas implementaciones de interceptors,
 	por ejemplo, \code{FieldInterceptor}, \code{MethodInterceptor}, ect. 
 	Cada \code{Interceptor} tiene un conjunto de \code{Behaviors}.
 	Los \code{Behavior} son los objetos que tienen la modificación a realizar sobre las clases. 
 	Estas modificaciones son configuradas como funciones por el usuario programador.
 	Existen varias estrategias de modificación siguiendo la teoria de AOP, \code{Before}, \code{After}, \code{Arround}, \code{ReadField} y \code{WriteField}. 
	 
	Finalmente una instancia de \code{APOClassLoader}, instalada como \emph{class
	loader} del sistema permite que antes de utilizar cualquier clase, esta pueda
	ser procesada por el \code{AdviceWeaver}.
	
	Todas las modificaciones configuradas están expresadas en un
	lenguaje de alto nivel y las traduce al lenguaje de bajo nivel que requiere el
	framework Javassist.
	La Figura \ref{pooCode} muestra un ejemplo de código de este lenguaje de alto nivel,
	tomado del framework POO, que se describe en la Sección \ref{poo}.
	
	\begin{figure}[h]
		\centering
		\includegraphics[width=450px, height=377px]{img/apo}
		\caption{Diagrama UML de la herramienta APO}
		\label{aopImage}
	\end{figure}	 
	
	
	A su vez, la Tabla \ref{table} describe las expresiones propias del lenguaje
	definido por APO, su traducción al lenguaje de expresiones de
	Javassist y su significado.
	
	\begin{figure}[h]
		\begin{lstlisting}
$Object oldValue = $oldValue;
$originalAsigment;
$this.firePropertyChange('$fieldName', oldValue, $newValue);
		\end{lstlisting}
		\caption{Fragmento de código del framework POO}
		\label{pooCode}
	\end{figure}
	
	
	\begin{table*}[h]\centering
		\ra{1.3}
		\begin{tabular}{|+l^l^p{7cm}|}\toprule			
			\hline
			\rowstyle{\bfseries}%
				Expr. APO & Expr. Javassist & Significado \\
			\hline
				\$Object & java.lang.Object & El nombre completo de la clase Object \\
			\hline
				\$this & \$0 & El objeto receptor del mensaje.\\
			\hline
				\$newValue & \$1 & El primer parámetro del método. \\
			\hline
				\$oldValue &  \$0.getAtribute() & El valor del atributo antes de
			la asignación que está siendo modificada.\\
			\hline
				\$originalAsigment & \$0.atribute = \$1 & La asignación del atributo con el
			primer parámetro del método.\\
			\hline
				``\$fieldName'' & ``atribute'' & El nombre del atributo como un String.\\
			\hline
		\bottomrule
		\end{tabular} 
		\caption{Tabla de equivalencia de expresiones. ``atribute'' es el nombre del atributo propiamente dicho.}
		\label{table}
	\end{table*}
	
	APO es una herramienta abstracta, es decir, por si sola no modifica las clases, hay que configurarlo adecuadamente
	para optener el resultado deceado. POO y POT se contruyeron siguiendo esta filosofia de creacion de aspectos.

	
\subsection{Pure Object Transaction}
	\label{pot} 
	\emph{Pure Object Transaction} (POT) es la herramienta que implementa el
aspecto transaccional definido en la sección \ref{aspectoTransaccional}.
Está basada en una implementación anterior de Nicolás Passerini y Javier
Fernandes, que se actualizó para aprovechar el framework APO y facilitar su
integración con las demás herramientas desarrolladas.

\medskip
 
Este framework intercepta todas las lecturas y escrituras de los atributos de
un objeto, delegando tanto las lecturas como las escrituras al
\emph{administrador de las transacciones}.
A su vez, el administrador de transacciones asocia el pedido con un contexto
transaccional, que guarda los valores de los atributos de un objeto que fueron
modificados durante la transacción en una estructura de la forma
\code{[objeto, [nombre del atributo, valor]]}.
Cada contexto transaccional está asociado a un \emph{thread}. Esto
permite manejar la concurrencia en el acceso a la información de los objetos.
Para aplicarle este aspecto a una clase se utiliza la \emph{annotation} \lstinline|Transactional| como
se muestra en la Figura \ref{annoTransactional}.

\begin{figure}[hbt]
	\begin{lstlisting} 
		@Transactional
		public class Account {
		}
	\end{lstlisting}
	\caption{Annotation para aplicar el aspecto transaccional.}
	\label{annoTransactional}
\end{figure}
	
\medskip
 
La herramienta provee también soporte para transacciones anidadas.
Al momento de hacer el \emph{commit} en una transacción, los valores
contenidos en el contexto transaccional son impactados en la transacción
padre.
En caso de tratarse de una transacción de primer nivel, los cambios se impactan
en los objetos de dominio usando \emph{reflection}.
Esta forma de implementación permite que la identidad del objeto se
mantenga, ya que el objeto no se modifica ni se clona, sólo se intercepta el
acceso a sus atributos.

Otro agregado a la versión original es la intersección de las modificaciones 
a un objeto de tipo \lstinline|Collection|, por ejemplo agregar o quitar
objetos de una colección.
Esto presenta un desafío especial ya que habitualmente en los programas Java
se utilizan las implementaciones de colecciones provistas por el propio
lenguaje y no es posible aplicar aspectos sobre estas clases. 
En la nueva versión, este problema se resuelve reemplazando en forma
automática las colecciones del lenguaje Java por
implementaciones propias de las mismas interfaces.
La figura \ref{potuml} muestra esquemáticamente el diseño de la herramienta.

\begin{figure}[hbt]
	\centering
	\includegraphics[scale=0.4]{img/pot}
 	\caption{Esquema de la herramienta POT}
 	\label{potuml}
\end{figure}

	
\subsection{Pure Observable Objects}
	\label{poo}
	\subsection{Pure Observable Objects}
	\label{poo}
	\emph{Pure Observable Objects} (POO) es el framework que implementa el
	aspecto Observable planteado en la Sección \ref{aspectoObservable}.
	La implementación interna del aspecto agrega un
	atributo llamado \lstinline|changeSupport| del tipo
	\lstinline|PropertySupport| al objeto al que se le aplica el aspecto.
	\lstinline|PropertySupport| es una interfaz, la implementación concreta a
	utilizar se obtiene del el archivo de configuración.
	
	Para completar el objetivo se agregan los métodos 
	\lstinline|addPropertyChangeListener| y
	\lstinline|removePropertyChangeListener| que permiten agregar 
	y remover observadores, y \lstinline|firePropertyChange|
	que notifica a los observadores que un atributo ha cambiado.
	
	Para entender mejor el modelo de classes, la la figura \ref{fig:poo} muesta
	esquemáticamente el diseño de la herramienta.
	
	\begin{figure}
		\includegraphics[width=450px, height=200px]{img/poo}
	 	\label{fig:poo}
	 	\caption{}
	\end{figure}
	
	Para agregarle este aspecto a una clase se utiliza la \emph{Annotation}
	 \lstinline|Observable|, como
	se muestra en la siguiente porción de código:
	\begin{lstlisting} 
		@Observable
		public class Client{
		}
	\end{lstlisting}

	
\subsection{Integración de POT, POO y Arena}
	\subsection{Integración de POT, POO y Arena}
La integración entre Arena y POO se realizó construyendo una implementación de
\lstinline|PropertySupport| que dispara los eventos de acuerdo a los esperado
por el framework Arena.

\medskip
Por otro lado, la clase \lstinline|TransactionalDialog| permite integrar Arena y
POT.
Definir una ventana como una subclase de
\lstinline|TransactionalDialog| asocia automáticamente a esa ventana con un
contexto transaccional.
Al abrirse la ventana se efectúa la operación de \emph{beginTransaction}.
Luego, botones \emph{Aceptar} y \emph{Cancelar} (que por defecto son agregados
por la superclase) efectúan las acciones de \emph{commit} y
\emph{rollback}.

\medskip
En tercer lugar, como se explicó en la sección \ref{sec:Union}, para integrar
los dos aspectos entre sí se requiere filtrar los eventos disparados por los objetos de dominio, 
limitándolos a las ventanas que se encuentran dentro del mismo contexto
transaccional. 
Se implementaron tres niveles de aislamiento de los eventos:
\begin{description}
	\item[\emph{Fire All}] Todos los eventos disparados por el dominio son
	escuchados, sin importar si están en un transacción.

	\item[\emph{Fire Committed}] Solo se escucha los eventos de las transacciones
		comiteadas
	
	\item[\emph{Fire olnly in my transaction}] solo se escucha los eventos que
		ocurren dentro de su translación.
 \end{description}
 
\medskip
Finalmente, el framework se puede configurar para utilizar uno, otro o ambos
aspectos, según se requiera.
Los objetos pueden ser anotados con \emph{Observable} y
\emph{Transactional} como vimos previamente, 
o bien utilizar \emph{TransactionalAndObservable} que es una unión de ambas.

	\begin{lstlisting} 
		@TransactionalAndObservable
		public class Client{
		}
	\end{lstlisting}

\subsection{Otras mejoras al Arena}
	La integración se realizo con el lenguaje de programación Scala
	\cite{OderskySpoonVenners08}. Para llevar al cabo la integración fue necesario agregar algunas
	mejoras en Arena:
	\begin{description}

	  \item[Bindings anidados] Como se vio en la Sección \ref{binding},
		  el \emph{binding} es una conección de propiedades entre dos objetos. Con
		  esta idea se desarrolló un tipo de binding que permite conectar propiedades
		  anidadas entre dos objetos, por ejemplo,  la figura \ref{bindAnidado}

			\begin{figure}[h]
			\centering
					\begin{lstlisting}
						bindProperty("source.owner.name	") 
					\end{lstlisting}
			\caption{Ejemplo de binding de con la Clase Transaction}
			\label{bindAnidado}
		\end{figure}	

	  \item[Monitor de Transacciones]
		 Se desarrolló un \emph{Monitor de Transacciones}, que permite
		 \emph{debuggear} las transacciones abiertas actualmente, mostrando
		 los objetos afectados por la transacción y los atributos que se
		 modificaron.
		La figura \ref{monitor} muestra el monitor de transacciones\ldots
		
		\begin{figure}[h]
			\centering
			\includegraphics[width=350px, height=380px]{img/monitorTransacciones.png}
			\caption{Monitor}
			\label{monitor}
		\end{figure}	
	
	  \item[Nuevos componentes] Se agregaron algunas estructuras visuales como
	  arboles y listas.
	\end{description}

	
\section{Aplicación de ejemplo}
	\label{Example}	
	\section{Aplicación de ejemplo}

Para comprender mejor esta problemática veamos un ejemplo. Supongamos una simple
aplicación, donde los clientes de un banco pueden transferir dinero de
una cuenta a otra. Al realizar una transferencia, hay que extraer el monto
indicado de una cuenta, y depositarlo en otra. 
Al ejecutar cualquiera de estas dos operaciones se pueden producir errores,
como por ejemplo, que no haya saldo saldo suficiente, o que el depósito supere
el máximo permitido.

La figura \ref{example} muestra el diagrama de clases de la aplicación de
ejemplo

	\medskip

	\begin{figure}[h]
		\centering
		\includegraphics[width=250px, height=150px]{img/transaccion}
		\caption{Diagrama UML de la aplicación de ejemplo}
		\label{example}
	\end{figure}	
	
La aplicación de ejemplo tiene los siguientes casos de uso:

\begin{enumerate}
  \item{\bf{Transferir Dinero}} 
	Tres ventajas:
	- queda más simple
	- menos posibilidad de mandármela
	- concurrencia.
  	
  \item{\bf{Transferencias múltiples}}
	muchas transferencias en una única transacción\ldots fijate si sale algo y si no
	lo volamos.
\end{enumerate}

La figura \ref{executeTransaction} muestra el método  \lstinline|execute| de la
clase \lstinline|Transaction| \begin{figure}[h]
	\begin{lstlisting}
		public void execute(){
			this.source.withdraw(this.amount);
			this.destination.deposit(this.amount);
		}
	\end{lstlisting}
	\caption{Fragmento de código de la Clase Transaction}
	\label{executeTransaction}
\end{figure}
