\section{Objetivo}
\label{sec:Objetivo}
El primer objetivo de este trabajo es automatizar la comunicación entre el
modelo y la interfaz de usuario.
Toda vez que el usuario ingrese información en el sistema, la
información debe llegar directamente hasta el dominio. Esto nos permite
aprovechar toda la lógica del dominio durante la edición.
De la misma forma, si cambia un atributo de un objeto de dominio, se debe
informar a la UI para que ese cambio sea visible
inmediatamente.
Toda esta funcionalidad debe poder ser implementada en forma transparente,
sin modificar el dominio ni imponerle restricciones.

El segundo objetivo es proveer un soporte para transacciones, que
permita automatizar el \emph{rollback} de que la operación en curso no se pueda
finalizar o que el usuario decida cancelarla.
Además, dos usuarios deberán poder trabajar con la misma información
simultáneamente con un grado de aislamiento \emph{read committed}.

El tercer objetivo es proveer una sistema de monitoreo para las transacciones,
que permita ver el estado de todas las transacciones actuales, y una descripción
sobre los objetos y los fields que están afectados por la transacción.
