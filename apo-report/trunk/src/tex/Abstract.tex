\begin{abstract}

/*Vamos tratar 2 temas especificos,*/

\begin {itemize}

\item {\bf Objetos Puros Transaccionales}\\

\item {\bf Objetos puros Observables}\\

\end{itemize}


{\bf Objetos Puros Transaccionales}\\
Investiga un particular enfoque de dise�o para hacer
frente a los sistemas transaccionales en un entorno orientado a objetos.
Este arquitectura es �nica debido a los siguientes objetivos de dise�o:

\begin {itemize}

\item {\bf Centrada en objetos puros}

\item {\bf Transparente}

\item {\bf Sistema independiente de la arquitectura} se puede utilizar e
integar en cualquier arquitectura del sistema: web, rich-client,
etc, y con cualquier sistema de almacenamiento persitence: RDBM, OODB, etc

\end {itemize}

{\bf Objetos Puros Observables}

Del tiempo necesario para desarrollar una interfaz de usuario, gran
parte se destina a la tarea rutinaria de traspasar los datos de los
objetos de dominio hacia los componentes de la interfaz gr�fica y
viceversa. A este proceso de vincular y convertir los datos entre la
interfaz y el dominio se lo denomina binding.
Otro problema que aparece frecuentemente es que el binding en su versi�n
m�s sencilla modifica los objetos directamente, por lo que al cancelar
una operaci�n se debe volver al estado anterior, y este proceso es
repetitivo y propenso a errores. Con frecuencia implica introducir
comportamiento propio de la interfaz de usuario, en mis objetos de
dominio, mezclando la l�gica de las dos partes de la aplicaci�n.
Esto me obliga a adaptar mi dominio a este sistema, y repetir muchas
l�neas de c�digo.


\end{abstract}


