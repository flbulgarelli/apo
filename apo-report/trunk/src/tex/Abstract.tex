\begin{abstract}

Al desarrollar interfaces de usuario utilizando la programación orientada a
objetos, con frecuencia gran parte del tiempo se destina a pocas tareas
rutinarias, como el traspaso de datos entre los objetos del dominio y los
componentes de la interfaz gráfica. Si bien esta tarea es simple, cuando su
realización es manual se vuelve propensa a errores.
Adicionalmente, muchas interfaces de usuario requieren
que el usuario tenga la posibilidad de cancelar la operación que está
realizando. En una aplicación multiusuario, las tareas para garantizar
consistencia ante una cancelación no son sencillas. En este trabajo se propone
una posible solución a estos problemas basada en programación orientada a
aspectos. Un primer aspecto permite que los objetos de dominio disparen eventos
en forma transparente al ser modificados, de forma de actualizar la interfaz de
usuario (UI) acorde a esos cambios. Un segundo aspecto automatiza las
operatoria relacionada con las cancelaciones, permitiendo que las
modificaciones a los objetos del dominio se realicen siguiendo el concepto de
transacción, poniendo el foco en las propiedades de atomicidad y consistencia.

\end{abstract}
