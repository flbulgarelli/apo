\begin{abstract}


Al desarrollar interfaces de usuario, con frecuencia gran parte del tiempo se
destina pocas tareas rutinarias, como el traspaso de datos entre los objetos
del dominio y los componentes de la interfaz gráfica. Si bien esta tarea es
simple, cuando su realización es manual la vuelve propensa a errores.
Adicionalmente, muchas interfaces de usuario requieren la posibilidad de que el
usuario cancele la operación que está realizando. En una aplicación
multiusuario, las tareas para garantizar consistencia ante una cancelación no
son sencillas. En este trabajo se propone una posible solución a estos
problemas basada en programación orientada a aspectos. Un primer aspecto
permite que los objetos de dominio en forma transparente disparen eventos al
ser modificados, de forma de actualizar la interfaz de usuario (UI) acorde a
esos cambios. Un segundo aspecto automatiza las operatoria relacionada con las
cancelaciones, permitiendo que las modificaciones a los objetos del dominio se
realicen siguiendo el concepto de transacción, y concentrándonos en las
propiedades de atomicidad y consistencia.

\end{abstract}


