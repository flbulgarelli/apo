\subsection{Pure Object Transaction}
	\label{pot} 
	Pure Object Transaction (POT) es la herramienta que implementa el aspecto
	transaccional como se definió en la sección \ref{aspectoTransaccional}.
	Está basado en una implementación anterior de Nicolás Passerini y Javier
	Fernandes, que se actualizó para aprovechar el framework APO y facilitar su
	integración con las demás herramientas desarrolladas.
	
	\medskip
	 
	Este framework intercepta todas las lecturas y escrituras de los atributos de
	un objeto, delegando tanto las lecturas como las escrituras al
	\emph{administrador de las transacciones}.
	A su vez, el administrador de transacciones asocia el pedido con un contexto
	transaccional, que guarda los valores de los atributos de un objeto que fueron
	modificador durante la transacción en una estructura de la forma
	\lstinline|[Objeto, [Atributo, Valor]]|.
	Cada contexto transaccional esta asociado a un \emph{thread}. Esto
	permite manejar la concurrencia en el acceso a la información de los objetos.
	La herramienta provee también soporte para transacciones anidadas.
	 
	Al momento de hacer el \emph{commit} en una transacción, los valores
	contenidos en el contexto transaccional son impactados en la transacción
	padre.
	En caso de tratarse de una trasacción de primer nivel, los cambios se impactan
	en los objetos de dominio.
	Esta forma de implementación permite que la identidad del objeto se
	mantenga, ya que el objeto no se modifica ni se clona, solo se intercepta el
	acceso a sus atributos.
	
	Para agregarle este aspecto a una clase se utiliza la \emph{Annotation}
	\emph{Transactional}.
			
		\begin{lstlisting} 
			@Transactional
			public class Client {
			}
		\end{lstlisting}
	
	\medskip
	
	Otro agregado a la versión original es la intercepción de las modificaciones 
	a un objeto de tipo \lstinline|Collection|, por ejemplo agregar o quitar
	objetos de una colección.
	Esto presenta un desafío especial ya que habitualmente en los programas Java
	se utilizan las implementaciones de colecciones provistas por el propio
	lenguaje y no es posible aplicar aspectos sobre estas clases. 
	En la nueva versión, este problema se resuelve reemplazando en forma
	automática las colecciones del lenguaje Java por
	implementaciones propias de las mismas interfaces.
