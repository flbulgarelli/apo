\section{Aplicación de ejemplo}
Para ilustrar el uso de las herramientas desarrolladas utilizaremos como ejemplo
una aplicación bancaria, en la que los clientes de un banco pueden transferir
dinero de una cuenta a otra. 
Realizar una transferencia implica extraer el
monto indicado de una cuenta, y depositarlo en otra. 
Al ejecutar cualquiera de estas dos operaciones se pueden producir errores,
como por ejemplo, que no haya saldo saldo suficiente, o que el depósito supere
el máximo permitido.
La figura \ref{example} muestra las clases que implementan la lógica del
dominio.

	\begin{figure}[h!]
		\centering
		\includegraphics[width=450px, height=250px]{img/transaccion}
		\caption{Diagrama UML de la aplicación de ejemplo}
		\label{example}
	\end{figure}	

A continuación se describirán las dos pantallas más importantes de la
aplicación, que nos permitirán mostrar las diferentes utilidades brindadas por
nuestra propuesta.
 
\subsection{Transferencia simple}
	La primera pantalla que describe una transferencia \ref{trasferenciaSimple},
	pemite elegir una de las cuentas de un cliente, y otra cuenta que este dentro
	del sistema, y realizar una transferencia de la primera hacia la segunda con el
	monto indicado. Utilizar ambos aspectos se llegan a destacar tes ventajas:
	
	\begin{description}
		\item[Código simple] El código solo se concentra en lo importante, que es
		debitar y extraer el monto.
		La figura \ref{executeTransaction} muestra el método  \lstinline|execute| de la
		clase \lstinline|Transaction| 
		\begin{figure}[h]
			\begin{lstlisting}
				public void execute(){
					this.source.withdraw(this.amount);
					this.destination.deposit(this.amount);
				}
			\end{lstlisting}
			\caption{Fragmento de código de la Clase Transaction}
			\label{executeTransaction}
		\end{figure}
		 
		\item[Menos posibilidad de cometer un error] Dado que código de dominio es
		limpio, no hay comportamiento fuera de la logica de negocio que pueda provocar
		un error.
		\item[concurrencia]???
	\end{description}
	
	
	\begin{figure}[h]
		\centering
		\includegraphics[width=450px, height=375px]{img/simple-transferencia}
		\caption{Pantalla de transferencia}
		\label{trasferenciaSimple}
	\end{figure}
  	
\subsection{Transferencias múltiples}
	muchas transferencias en una única transacción\ldots fijate si sale algo y si no
	lo volamos.

