\subsection{Integración de POT, POO y Arena}
Para integar el framework Arena con el aspecto transaccional (POT) se definió
la clase \lstinline|TransactionalDialog|. 
Definir una ventana como una subclase de
\lstinline|TransactionalDialog| permite asociar a esa ventana con un contexto
transaccional.
Al abrirse la ventana se efectúa la operación de \emph{beginTransaction}.
Luego, botones \emph{Aceptar} y \emph{Cancelar} (que por defecto son agregados
por la superclase) efectúan las acciones de \emph{commit} y
\emph{rollback}.

Como se explicó en la sección \ref{sec:Union}, para integrar los dos aspectos
entre sí se requiere filtrar los eventos disparados por los objetos de dominio, 
limitándolos a las ventanas que se encuentran dentro del mismo contexto
transaccional. 
Se implementaron tres niveles de aislamiento de los eventos:
\begin{description}
	\item[\emph{Fire All}] Todos los eventos disparados por el dominio son
	escuchados, sin importar si están en un transacción.

	\item[\emph{Fire Committed}] Solo se escucha los eventos de las transacciones
		comiteadas
	
	\item[\emph{Fire olnly in my transaction}] solo se escucha los eventos que
		ocurren dentro de su translación.
 \end{description}
 
El framework se puede configurar para utilizar uno, otro o ambos
aspectos, según se requiera.
Los objetos pueden ser anotados con \emph{Observable} y
\emph{Transactional} como vimos previamente, 
o bien utilizar \emph{TransactionalAndObservable} que es una union de ambas.

	\begin{lstlisting} 
		@TransactionalAndObservable
		public class Client{
		}
	\end{lstlisting}

\subsection{Otras mejoras al Arena}
	La integración se realizo con el lenguaje de programación Scala
	\cite{???}. Para llevar al cabo la integración se agregó algunas
	mejoras en el Arena. 
	Algunas mejoras fueron:
	\begin{description}
	  \item[Monitor de Transacciones]
		 Con arena también se desarrolló un \emph{Monitor de Transacciones}, que 
		 muestra el estado actual de la transacción, incluyendo las transacciones
		 anidadas. Mostrando los objetos afectados por la transacción y los
		 atributos se modificaron.
	  \item[Nuevos componentes] Se Implementaron algunas estructuras visuales como
	  Arboles y Listas.
	  \item[Bindinds Anidados] Se implemento bindings para las propiedades
	  anidadas de los objetos.
	\end{description}
