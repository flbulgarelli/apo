La integración entre Arena y POO se realizó construyendo una implementación de
\lstinline|PropertySupport| que dispara los eventos de acuerdo a los esperados
por el framework Arena.

\medskip
Por otro lado, la clase \lstinline|TransactionalDialog| permite integrar Arena y
POT, dado que al definir una ventana como una subclase de
\lstinline|TransactionalDialog|, ésta se asocia automáticamente con un
contexto transaccional.
Al abrirse la ventana se efectúa la operación de \emph{beginTransaction}.
Luego, botones \emph{Aceptar} y \emph{Cancelar} (que por defecto son agregados
por la superclase) efectúan las acciones de \emph{commit} y
\emph{rollback}.

\medskip
En tercer lugar, como se explicó en la sección \ref{sec:Union}, para integrar
los dos aspectos entre sí se requiere filtrar los eventos disparados por los objetos de dominio, 
limitándolos a las ventanas que se encuentran dentro del mismo contexto
transaccional. 
Se implementaron tres niveles de aislamiento de los eventos:
\begin{description}
	\item[\emph{Fire all}] Todos los eventos disparados por el dominio son
	escuchados, sin importar si están en un transacción.

	\item[\emph{Fire committed}] Sólo se escucha los eventos de las transacciones
		comiteadas
	
	\item[\emph{Fire only in my transaction}] sólo se escucha los eventos que
		ocurren dentro de su transacción.
 \end{description}
 
\medskip
Finalmente, el framework se puede configurar para utilizar uno, otro o ambos
aspectos, según se requiera.
Los objetos pueden ser anotados con \emph{Observable} y
\emph{Transactional} como vimos previamente, 
o bien utilizar \emph{TransactionalAndObservable} que es una unión de ambas como se muestra en la Figura \ref{taov}.

	\begin{figure}[hbt]
		\begin{lstlisting} 
			@TransactionalAndObservable
			public class Account{
			}
		\end{lstlisting}
		\caption{Annotation para los aspectos Observable y Transaccional}
		\label{taov}
	\end{figure}


\subsection{Otras mejoras al Arena}
	La integración se realizó con el lenguaje de programación Scala
	\cite{OderskySpoonVenners08}. Para llevar a cabo la integración fue necesario
	agregar las siguientes mejoras en el framework Arena:
	\begin{description}

	  \item[Bindings anidados] Como se vio en la Sección \ref{binding},
		  el \emph{binding} es una conexión de propiedades entre dos objetos. Con
		  esta idea se desarrolló un tipo de binding que permite conectar propiedades
		  anidadas entre dos objetos. 
		  
		  La figura \ref{bindAnidado} muestra un ejemplo de binding anidado.

		\begin{figure}[hbt]
			\centering
					\begin{lstlisting}
						bindProperty("customer.address.street") 
					\end{lstlisting}
			\caption{Ejemplo de binding anidado de la clase Transaction}
			\label{bindAnidado}
		\end{figure}	

	  \item[Nuevos componentes] Se agregaron estructuras visuales como árboles y listas.

  	  \item[Monitor de transacciones]
		 Se desarrolló un \emph{Monitor de Transacciones}, que permite
		 \emph{debuggear} las transacciones abiertas actualmente, mostrando
		 los objetos afectados por la transacción y los atributos que se
		 modificaron.
		La figura \ref{monitor} muestra la pantalla del monitor de transacciones.
		
		\begin{figure}[hbt]
			\centering
			\includegraphics[scale=0.5]{img/monitorTransacciones.png}
			\caption{Monitor de transacciones}
			\label{monitor}
		\end{figure}	
	\end{description}
