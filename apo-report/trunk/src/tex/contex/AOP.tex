\subsection{Programación orientada a Aspectos}

	La programación orientada a aspectos (AOP) es un paradigma de programación cuya
	intención es mejorar la modularización de las aplicaciones.
	En este paradigma se considera que en una aplicación existen problemas 
	que se deben solucionar que son ortogonales a la lógica del dominio \cite{Kicz97a}.
	Mientras que en otros paradigmas el código necesario para solucionar estos problemas 
	se disemina por la estructura de las unidades funcionales,
	el paradigma AOP provee herramientas para atacar estos problemas sin modificar el código del dominio.

	\bigskip
	
	Un \emph{aspecto} es la unidad básica de construcción en AOP, debido a que permite
	modularizar los conceptos transversales (\emph{cross-cutting concerns}) presentes en una aplicación.
	Por ejemplo \emph{Logging}, \emph{Profiling}, entre otros.
	La definición de un aspecto se basa en los conceptos de: {\bf \emph{Join
	Point}}, {\bf \emph{ Point Cut}} y {\bf \emph{ Advice}}.
	
	\begin{quote}
	
	\begin{description}
		\item[\emph{Join Point}] Un \emph{Join Point} puede ser definido como un punto
		de interés en el código, como por ejemplo: la instanciación de una clase, el manejo de una
		excepción, una llamada a un método, el retorno de un método, la asignación a variable de instancia, etc.
		
		\item[\emph{Point Cut}] Un \emph{Point Cut} es un predicado o condición que selecciona un conjunto 
		de join points.
		
		\item[\emph{Advices}] Los \emph{Advices} son las acciones que se ejecutan en cada \emph{Join Point} dentro de un mismo
		\emph{Point Cut}, estas acciones se traducen en rutinas o fragmentos de código.
	
	\end{description}
	\end{quote}
