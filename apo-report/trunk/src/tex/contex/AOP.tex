\subsection{Programación orientada a Aspectos}

	La programación orientada a aspectos (POA) es un estilo de programación cuyo
	principal objetivo es mejorar la  separación entre los requerimientos
	funcionales de los no funcionales, para ganar expresividad, disminuyendo la
	dispersión del código asociado a un concepto y haciendo que las
	implementaciones resulten más comprensibles, adaptables y reutilizables.


\subsubsection{Definición de un Aspecto}
	Según el trabajo seminal de Gregor Kiczales y otros, un aspecto es 
	``una unidad modular que se disemina por la estructura de
	otras unidades funcionales. Los aspectos existen tanto en la etapa de
	diseño como en la etapa de implementación. Un aspecto de diseño es
	una unidad modular que se entremezcla en la estructura de otras partes
	del diseño. Un aspecto de programa o de código es una unidad modular
	del programa que aparece en otra unidades del programa.'' \cite{Kicz97a}

	\bigskip
	
	Un aspecto es la unidad básica de construcción en la POA, debido a que permite
	modularizar los conceptos transversales o \emph{cross-cutting concern} presentes en una aplicación.
	
	En palabras simples un aspecto puede ser definido como: ``La encapsulación y
	modularización de un \emph{cross-cutting concern}''. Por ejemplo
	\emph{Loggin}, \emph{Profiling}, entre otros.
	
	
	La definición de un aspecto se basa en los conceptos de: {\bf \emph{Join
	Point}}, {\bf \emph{ Point Cut}} y {\bf \emph{ Advice}}.
	
	\begin{quote}
	
	\begin{description}
		\item[\emph{Join Point}] Un \emph{Join Point} puede ser definido como un punto
		en la ejecución de una aplicación, como por ejemplo: la creación de una instancia, el manejo de una
		excepción, una llamada a un método, el retorno de un método, la asignación de
		un valor a una variable, etc.
		
		\item[\emph{Point Cut}] Un \emph{Point Cut} hace referencia a un conjunto de
		Join Point que cumplen cierta condición, es decir, permiten exponer el contexto de
		ejecución de dichos \emph{points}.
		
		\item[\emph{Advices}] Y por último los \emph{Advices}, éstos pueden definirse
		como: acciones que se ejecutan en cada \emph{Join Point} dentro de un mismo
		\emph{Point Cut}, estas acciones se traducen en rutinas o fragmentos de
		código.
	
	\end{description}
	\end{quote}
	
	
