\subsection{Transacciones}	

Una transacción es una unidad en la ejecución de un programa que se comporta 
\emph{atómicamente}: debe ejecutarse por completo o fallar sin dejar rastro, no
puede dejar los datos en un estado intermedio.

Los objetivos de trabajar con transacciones son dos.
En primer lugar, definir unidades de trabajo seguras, que permitan 
recuperarse de los errores y mantener la coherencia de los datos, incluso en caso de fallas en el sistema.
En segundo lugar, permitir el acceso concurrente, es decir, que múltiples
usuarios utilicen la misma información al mismo tiempo sin interferir entre sí.

\medskip
Para poder trabajar con transacciones se deben contar con instrucciones
especiales para marcar su inicio y su final:
\begin{quote}
	\label{ctxTransactional}
	\begin{description}
		\item[Begin transaction] es una operación que marca el inicio de una
		transacción. Toda operación que se haga de aquí en más quedará en suspenso
		mientras dure la transacción.
	
		\item[Commit] es una operación que finaliza y confirma los
		cambios realizados desde el inicio de la transacción. 
		
		\item[Rollback] es una operación que finaliza y revierte todos los cambios
		realizados desde el inicio de la transacción, 
		dejando los datos en el estado que tenían antes de comenzar.
	\end{description}
\end{quote}
   
\bigskip

Muchas veces las transacciones se asocian con los sistemas de gestión de bases
de datos, sin embargo es un concepto más general.
Por ejemplo, en interfaces de usuario es común incluir botones de
``aceptar'' y ``cancelar'' que podemos relacionar con las ideas de
\emph{commit} y \emph{rollback}. 
Si el usuario presiona el botón ``cancelar'' lo que se espera
es que los datos de la aplicación queden en el estado exacto en el que estaba
antes de comenzar la tarea.

\bigskip
\label{sec:ACID}
Para garantizar la integridad de los datos, se considera que un sistema de
transaccional debe cumplir con un conjunto de propiedades denominadas por el
acrónimo \emph{ACID} \cite{HaerderReuter83}.
De las propiedades ACID, este trabajo se concentra en las de \emph{Atomicidad},
\emph{Consistencia} y \emph{Aislamiento}. 

\begin{description}
	\item[Atomicidad]
		Una operación en un sistema de software se considera \emph{atómica} cuando se
		puede garantizar que no se realizará parcialmente aún cuando ocurran errores
		durante su ejecución. 
		En caso de operaciones complejas que constan de varias sub-operaciones, si una
		de las sub-operaciones no se puede completar, ninguna de las sub-operaciones
		debe ejecutarse. Si se detecta una falla después de que una de las
		sub-operaciones ya fue realizada, se debe deshacer dicha
		sub-operación, garantizando que el sistema vuelva al estado original antes de
		comenzar la operación.
		
	\item[Consistencia]
		Un sistema transaccional se considera que cumple con la propiedad de
		\emph{consistencia} si existe una forma de garantizar la integridad de los
		datos cada vez que termina una transacción.
		En programación orientada a objetos, el encapsulamiento permite que cada
		objeto se ocupe de su propia consistencia. 
		Entonces, una estrategia para lograr consistencia es evitar que otras partes
		de la aplicación manipulen los datos del dominio por fuera de los
		propios objetos de dominio.
	  
	\item[Aislamiento] \label{isolation}
		Para cumplir la propiedad de \emph{aislamiento}, ninguna
		transacción debe interferir con la ejecución de otra transacción.
		Es una práctica común considerar diferentes \emph{niveles de aislamiento}
		\cite{ANSI:1992:ANSd}, que miden el grado de independencia que tienen dos
		transacciones ejecutándose al mismo tiempo.

		Las técnicas propuestas en este trabajo intentarán alcanzar el nivel
		denominado \emph{Read committed}.
		En este nivel solo se pueden ver los datos que están confirmados, es decir, si
		una transacción esta haciendo una modificación, esta no es visible a otra
		transacción hasta que sea confirmada.

% 	\item{\bf \emph{Read uncommitted}}
% 				Este es el menor nivel de aislamiento. En él se permiten las lecturas
% 				sucias, es decir, una transacción pude ver cambios no confirmados
% 				aún por otra transacción.]

% 		A continuación se listan los niveles de aislamiento que presentan mayor
% 		interés en este trabajo:
% 		
% 		\begin{description}
% 			\item []
% 				
% 		  \item [Serializable]
% 				Este es el nivel de aislamiento más alto. Especifica que todas las
% 				transacciones ocurran de modo aislado, o dicho de otro modo, como si todas las
% 				transacciones se ejecutaran secuencialmente, es decir, una tras otra. 
% 			\end{description}
	
	\end{description}
		
