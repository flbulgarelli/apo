\subsection{Transacciones:}	

Una Transacción es un unidad de la ejecución de un programa, realizada dentro de
un sistema de gestión de base de datos (SGBD),  que accede y, posiblemente,
actualiza varios elementos de datos.

Una Transacción está delimitada por instrucciones de inicio transacción y fin
transacción. La transacción consiste en todas las operaciones que se ejecutan
entre inicio transacción y fin transacción.

Las transacciones en un entorno de base de datos tienen dos propósitos
principales:

\begin {enumerate}
  \item Definir unidades de trabajo seguras, que permitan la recuperación
  correcta de los errores y mantener una base de datos coherente, incluso en
  los casos de fallo del sistema, cuando la ejecución se detiene (total o
  parcialmente) y muchas operaciones sobre una base de datos siguen siendo
  incompletos..
  
  \item Permitir el acceso concurrente a los datos.
   
\end{enumerate}

\subsubsection{Propiedades ACID}
\label{sec:ACID}
Se denomina ACID a las propiedades de las transacciones que debe mantener el
SGBD para garantizar la integridad de los datos. \cite{HaerderReuter83}.

	\begin{description}
	\item[Atomicidad]
		La atomicidad asegura que se realizan todas las operaciones de una transacción
		o no se realiza ninguna.
		
	\item[Consistencia]
		 La consistencia asegura que una transacción no romperá con la integridad de
		 los de datos.
	  
	\item[Aislamiento]	
		El aislamiento asegura que ninguna transacción debe interferir con la
		ejecución de otra transacción.
		
		Un SGBD generalmente hace un bloqueo de los datos o implementa un Control de
		concurrencia mediante versiones múltiples, lo que puede resultar en una
		pérdida de concurrencia. Por ello se necesita añadir lógica adicional al
		programa que accede a los datos para su funcionamiento correcto.
		La mayor parte de los SGBD ofrecen unos ciertos \emph{niveles de aislamiento}
		que controlan el grado de bloqueo durante el acceso a los datos.
		
		Los niveles de aislamiento están definidos por \emph{ANSI/ISO SQL}, y se listan
		a continuación.
		
		\begin {itemize}
		
			\item{\bf \emph{Read uncommitted}}
				Este es el menor nivel de aislamiento. En él se permiten las lecturas
				sucias, es decir, una transacción pude ver cambios no confirmados
				aún por otra transacción.
			
			\item {\bf \emph{Read committed}}
				En este nivel solo se pueden ver los datos que están confirmados, es decir si
				una  transacción esta haciendo una modificación, esta no es visible a otra
				transacción hasta que sea confirmada.
				
			\item {\bf \emph{Repeatable reads}} 
				En este nivel de aislamiento, un SGBD que implemente el control de
				concurrencia basado en bloqueos, mantiene los bloqueos de lectura y escritura
				-de los datos seleccionados- hasta el final de la transacción. Sin embargo, no
				se gestionan los bloqueos de rango, por lo que las lecturas fantasma pueden
				ocurrir.
			
				
		  \item {\bf \emph{Serializable}}
				Este es el nivel de aislamiento más alto. Especifica que todas las
				transacciones ocurran de modo aislado, o dicho de otro modo, como si todas las
				transacciones se ejecutaran secuencialmente, es decir, una tras otra. 
			
			\end{itemize}
	  	
	\item[Durabilidad]
		
		Durabilidad significa que una vez que una transacción ha sido comiteada, ésta
		persistirá sus cambios, incluso si el sistema falla.
	
	\end{description}
	
\subsubsection{Operaciones}
	Las transacciones definen un conjunto de operaciones generales para su
	utilización:
	
	\begin{quote}
		\begin{description}
				\item[Commit] es una operación que finaliza  y confirma los
				cambios realizados dentro de una transacción. 
				
				\item[Rollback] es una operación que finaliza y revierte todos los cambios
				realizados dentro de una transacción.
				 
		\end{description}
	\end{quote}
	
\subsubsection{Aplicabilidad}
	Si bien la palabra transacción aparece fuertemente relacionada a la
	transacciones de SGBD, el concepto de una actividad ``transaccional'' es
	aplicable, prácticamente a todo contexto del desarrollo de software, es decir,
	las propiedades de atomicidad, consistencia, aislacion y durabilidad, son
	deseables en el desarrollo de software en todo momento. Por ejemplo, es
	habitual que las interfaces de usuario incluyan botones de aceptar y cancelar
	que son asimilables a las operaciones de \emph{commit} y \emph{rollback}
	respectivamente; Si el usuario presiona el botón ``cancelar'' lo que se espera
	es que los datos de la aplicación queden inalterados en el estado que estaba al
	realizar la tarea.
