\subsection{Transacciones}	

Una transacción es una unidad en la ejecución de un programa que se comporta 
\emph{atómicamente}: debe ejecutarse por completo o fallar sin dejar rastro, no
puede dejar los datos en un estado intermedio.
Los objetivos de trabajar con transacciones son dos.
En primer lugar, definir unidades de trabajo seguras, que permitan la
recuperarse de los errores y mantener la coherencia de los datos, incluso en
los casos de fallas en el sistema.
En segundo lugar, permitir el acceso concurrente, es decir, que múltiples
usuarios utilicen la misma información al mismo tiempo sin interferir entre sí.

Para poder trabajar con transacciones se deben contar con instrucciones
especiales para marcar su inicio y su final:
\begin{quote}
	\label{ctxTransactional}
	\begin{description}
		\item[Begin transaction] es una operación que marca el inicio de una
		transacción. Toda operación que se haga de aquí en más quedará en suspenso
		mientras dure la transacción.
	
		\item[Commit] es una operación que finaliza y confirma los
		cambios realizados desde el inicio de la transacción. 
		
		\item[Rollback] es una operación que finaliza y revierte todos los cambios
		realizados desde el inicio de la transacción.
	\end{description}
\end{quote}
   
\bigskip

Muchas veces las transacciones se asocian con los sistemas de gestión de bases
de datos (SGBD), sin embargo es un concepto más general.
En interfaces de usuario es común incluir botones de
``aceptar'' y ``cancelar'' que podemos relacionar con las ideas de
\emph{commit} y \emph{rollback}. 
Si el usuario presiona el botón ``cancelar'' lo que se espera
es que los datos de la aplicación queden en el estado exacto en el que estaba
antes de comenzar la tarea.

\bigskip

\subsubsection{Propiedades ACID}
\label{sec:ACID}
Se denomina ACID a las propiedades de las transacciones que debe mantener el
SGBD para garantizar la integridad de los datos. \cite{HaerderReuter83}.

	\begin{description}
	\item[Atomicidad]
		La atomicidad asegura que se realizan todas las operaciones de una transacción
		o no se realiza ninguna.
		
	\item[Consistencia]
		 La consistencia asegura que una transacción no romperá con la integridad de
		 los de datos.
	  
	\item[Aislamiento]	
		El aislamiento asegura que ninguna transacción debe interferir con la
		ejecución de otra transacción.
		
		Un SGBD generalmente hace un bloqueo de los datos o implementa un Control de
		concurrencia mediante versiones múltiples, lo que puede resultar en una
		pérdida de concurrencia. Por ello se necesita añadir lógica adicional al
		programa que accede a los datos para su funcionamiento correcto.
		La mayor parte de los SGBD ofrecen unos ciertos \emph{niveles de aislamiento}
		que controlan el grado de bloqueo durante el acceso a los datos.
		
		Los niveles de aislamiento están definidos por \emph{ANSI/ISO SQL}, y se listan
		a continuación.
		
		\subsubsection{Niveles de aislamiento}
	
		\begin{description}

			\item [Read uncommitted]
				Este es el menor nivel de aislamiento. En él se permiten las lecturas
				sucias, es decir, una transacción pude ver cambios no confirmados
				aún por otra transacción.
			
			\item [Read committed]
				En este nivel solo se pueden ver los datos que están confirmados, es decir si
				una  transacción esta haciendo una modificación, esta no es visible a otra
				transacción hasta que sea confirmada.
				
			\item [Repeatable reads] 
				En este nivel de aislamiento, un SGBD que implemente el control de
				concurrencia basado en bloqueos, mantiene los bloqueos de lectura y escritura
				-de los datos seleccionados- hasta el final de la transacción. Sin embargo, no
				se gestionan los bloqueos de rango, por lo que las lecturas fantasma pueden
				ocurrir.
			
				
		  \item [Serializable]
				Este es el nivel de aislamiento más alto. Especifica que todas las
				transacciones ocurran de modo aislado, o dicho de otro modo, como si todas las
				transacciones se ejecutaran secuencialmente, es decir, una tras otra. 
			
			\end{description}
	  	
	\item[Durabilidad]
		
		Durabilidad significa que una vez que una transacción ha sido comiteada, ésta
		persistirá sus cambios, incluso si el sistema falla.
	
	\end{description}
		
