
\section{Trabajo futuro}
\label{sec:futurework}


	\subsection{Estrategias de resolución de conflictos en las Transacciones}
	
		\begin{itemize}
	
			\item{\bf Optimista:} Que las múltiples  transacciones simultaneas puedan ser
			mejorado con un simple feedback y la información de alerta al
			usuario final diciendo que alguien más está editando el mismo objeto.
			Se podría ser más especifico, si se aumenta el nivel de detalle que especifica
			``quien'' y ``que objetos'' se comparten y ``cual de ellos se ha cambiado e
			otra operación ''
		
			\item{\bf Pesimista:} Se  bloquea el casos de uso, cuando alguien ya esta
			manipulando uno de los objetos involucrados. 
	
		\end{itemize}
		
	\subsection{Niveles de Aislamiento en las Transacciones}
	Actualmente solo soporta el nivel \emph{Read committed}, estaria bueno
	implementar los demás niveles de aislamiento \ref{isolation}.
	
	\subsection{Aspectear Arrays}
		Actualmente los cambios realizados en las propiedades de los objetos como
		Arrays, Listas, Mapas, ect, no se aplican el aspecto de Observabilidad cuando
		se operan con ellos. Por ejemplo en las operaciones de \emph{add}, \emph{remove} o
		equivalentes, no estamos realizando una nueva asignación del atributo, sino
		que internamente la estructura se modifica, entonces el objeto de dominio no
		tira eventos de cambio y por ende la interfaz no se actualiza.
		
	\subsection{Mejorar APO}
		Se podrian realizar mejoras en la herramienta para brindar mas usabilidad al
		momento de crear los aspectos. como por ejemplo brindar cpnfiguraciones para
		interceptar métodos, constructores, instanciaciones, excepciones, ect.
		
	\subsection{Monitoreo y debuggin de transacciones}
		Proveer una interfaz para monitoreo remoto en tiempo real. Podrian ser plugins
		de eclipse que permitan ver las transacciones activas. 
		Poder browsear los objetos modificados, comparar contra diferentes valores en otras transac-
		ciones. Un grafo de transacciones, etc. 
		Estadisticas de los casos de uso mas utilizados, los
		objetos mas editados, modificados, creados, borrados. 
		Estadisticas de uso por usuarios, etc. Se podria
		utilizar para entender cada usuario que es lo que mas
		utiliza del sistema, o identificar roles.

	
	\subsection{Integrar POO y POT independientemente del Arena}
		Actualmente la integración ambos aspectos se realizó utilizando conceptos
		implementados en Arena, ya que para la integración se utiliza una clase
		\lstinline|TransactionalObservableValue| que extiende de
		\lstinline|AbstractObservableValue|. \emph{TransactionalObservableValue}
		intercepta la notificación de cambio, evaluando según que
		\lstinline|IsolationLevelEvents| se esté utilizando en el momento.
		 
