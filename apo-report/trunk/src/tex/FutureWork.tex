\section{Trabajo futuro}
\label{Futurework}

\begin{description}

	\item[Estrategias de resolución de conflictos en las Transacciones]
		Actualmente el aspecto transaccional no cuenta con una estrategia de
		resolución de conflictos en	 el momento que dos o más transacciones
		confirman sus cambios sobre los mismos atributos de un objeto.
		Una estrategia posible es que se notifique al usuario que ese
		objeto esta siendo modificado por otra transacción.
		Otra estrategia podría ser que se bloquee el caso de uso cuando otra
		transacción este manipulando alguno de los objetos involucrados.

	\item[Niveles de Aislamiento en las Transacciones]
		Actualmente solo soporta el nivel \emph{Read committed}, sería conveniente
		implementar los demás niveles de aislamiento \ref{isolation}.
	
	\item[Aplicación de aspecto observable en las Colecciones]
		Actualmente, a los cambios realizados en las propiedades de los objetos
		como \lstinline|Arrays|, \lstinline|Listas|, \lstinline|Mapas|, etc no se les
		aplica el aspecto observable cuando se opera con ellos. Por ejemplo, las
		operaciones de \emph{add} o \emph{remove} de una colección no producen
		eventos.

	\item[Integración de POO y POT independientemente del Arena]
		Actualmente la integración de ambos aspectos se realizó utilizando conceptos
		implementados en Arena. Para llevar a cabo la integración se utilizó la clase
		\lstinline|TransactionalObservableValue| que extiende de
		\lstinline|AbstractObservableValue|. \emph{AbstractObservableValue} es una
		clase del framework que utiliza Arena para realizar el \emph{binding}.

\end{description}