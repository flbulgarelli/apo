
\section{Trabajo futuro}
\label{futurework}

	\subsection{Estrategias de resolución de conflictos en las Transacciones}
		Actualmente el aspecto transaccional no cuenta con una estrategia de
		resolucion de conflictos en	 el momento que dos o mas transacciones
		confirman sus cambios sobre los mismos atributos de un objeto.
		Una estrategia posible puede ser que se notifique al usuario que ese
		objeto esta siendo modificado por otra transaccion.
		Otra estrategia podria ser que se bloquee el caso de uso cuando otra
		transaccion este manipulando alguno de los objetos involucrados.

	\subsection{Niveles de Aislamiento en las Transacciones}
		Actualmente solo soporta el nivel \emph{Read committed}, estaría bueno
		implementar los demás niveles de aislamiento \ref{isolation}.
	
	\subsection{Aplicación de aspecto observable en las Colecciones}
		Actualmente los cambios realizados en las propiedades de los objetos como
		\lstinline|Arrays|, \lstinline|Listas|, \lstinline|Mapas|, etc, no se aplica
		el aspecto observable cuando se operan con ellos. Por ejemplo en las
		operaciones de \emph{add}, \emph{remove} o equivalentes, no se esta 
		realizando una nueva asignación del atributo, sino que internamente la
		estructura se modifica, entonces el objeto de dominio no tira eventos de
		cambio y por ende la interfaz no se actualiza.
		
	\subsection{Mejorar APO}
		Se podrían realizar mejoras en la herramienta para brindar mas usabilidad al
		momento de crear los aspectos. como por ejemplo brindar configuraciones para
		interceptar métodos, constructores, instanciaciones, excepciones, etc.
		
	\subsection{Integración de POO y POT independientemente del Arena}
		Actualmente la integración ambos aspectos se realizó utilizando conceptos
		implementados en Arena, ya que para la integración se utiliza una clase
		\lstinline|TransactionalObservableValue| que extiende de
		\lstinline|AbstractObservableValue|. \emph{AbstractObservableValue} es una
		clase de el framework que utiliza Arena para realizar el binding.
