\section{Introducción}

Este trabajo tiene por objetivo automatizar dos tareas comunes en el
desarrollo de aplicaciones orientadas a objetos.

La primera de estas tareas es el traspaso de datos entre los objetos del dominio
y los componentes de la vista, en el contexto de la creación de interfaces de
usuario siguiendo el patrón \emph{MVC}.
 
La segunda tarea es proveer un mecanismo que permita deshacer cambios que se
hubieran realizado como parte de una operación que se comenzó y que no se puede
o no se desea finalizar.
En el contexto de la programación orientada a objetos, estas operaciones son
modificaciones en el estado de un objeto.
Por ejemplo, si un usuario cancela una operación desde
la interfaz o si se produce una excepción durante la ejecución, la aplicación
debe garantizar que todos los objetos afectados se devuelvan al estado que
tenían antes de comenzar la operación inconclusa.

\medskip 

En este trabajo se propone una posible solución a estos problemas, basada en
\emph{programación orientada a aspectos}.
Un primer aspecto, llamado \emph{Observable} (POO), permite que, en forma
transparente, los objetos de dominio disparen eventos al ser modificados.
De esta forma la interfaz de usuario puede actualizarse y reflejar esos
cambios en forma automática.

Un segundo aspecto, llamado \emph{Transaccional} (POT), automatiza la
cancelación de una operación, garantizando que todos los objetos quedan en el
mismo estado en el que estaban antes de comenzar. 
Esto permite que las modificaciones a los objetos del dominio se realicen
en forma \emph{transaccional}, en particular las propiedades de
\emph{atomicidad}, \emph{consistencia} y \emph{aislamiento}.

Para desarrollar POT y POO se construyó una herramienta llamada
\emph{Aspects for Pure Objects} (APO).
APO es una herramienta para programación orientada a aspectos que abstrae
conceptos fundamentales y complejos del framework \emph{Javassist}. 
APO es independiente de POT y POO, se puede utilizar
para la creación de otros aspectos.

Para mostrar la posibilidad de componer los aspectos POO y POT se los integró en
una extensión del framework \emph{Arena}.
Arena es un framework educativo para la creación de interfaces de usuario
utilizado.
Las extensiones desarrolladas en el marco de este trabajo han sido incorporadas
a la última versión del Arena que ya está siendo utilizada en la materia de
Construcción de Interfaces de Usuario de la Universidad Nacional de Quilmes,
entre otras.

\bigskip

\subsection{Estructura de este trabajo}
\noindent Este trabajo consta de las siguientes secciones:

\begin{itemize}
	\item \textbf{Contexto}\\
		En la Sección \ref{Context} se introducen algunos de los conceptos básicos que
		son necesarios para la compresión del trabajo.
	\item \textbf{Estado del Arte}\\
		En la Sección \ref{StateOfTheArt} se describen las estrategias más comunes
		utilizadas actualmente en la industria para solucionar atacados por este trabajo.
	\item \textbf{Objetivo y Solución}\\
		La Sección \ref{Objective} detalla el objetivo de este trabajo y la Sección
		\ref{Solucion} describe la estrategia propuesta para alcanzarlo.
	\item \textbf{Implementación y Ejemplo}\\
		Las Secciones \ref{Implementacion} y \ref{Example} describen la
		implementación de la herramienta y proveen ejemplos de utilización de la misma.
	\item \textbf{Conclusiones y Trabajo futuro}\\
		En las Secciones \ref{Conclusions} y \ref{Futurework} se detallan las
		conclusiones del trabajo y y posibles caminos para continuarlo.
	\item \textbf{Configuración e instalación}\\
		La Sección \ref{Configuracion} provee las instrucciones necesarias para poder instalar 
		y utilizar las herramientas desarrolladas en este trabajo. 
\end{itemize}
