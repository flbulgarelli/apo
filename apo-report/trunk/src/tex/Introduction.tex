\section{Introducción}

Este trabajo tiene por objetivo automatizar dos tareas comunes en el
desarrollo de aplicaciones orientadas a objetos.

La primera de estas tareas es el traspaso de datos entre los objetos del dominio
y los componentes de la vista en el contexto de la creación de interfaces de
usuario siguiendo el patrón \emph{MVC}.
 
La segunda tarea es proveer un mecanismo que permita deshacer los cambios
realizado en el contexto de una tarea que no se desea o no es posible terminar.
Por ejemplo, cuando uno usuario cancela una operación desde la interfaz o bien
cuando una excepción impide terminar un procedimiento que ya comenzó a realizar
cambios.
En el contexto de la programacion orientada a objetos, estas operaciones son
modificaciones en el estado de un objeto.

%En una aplicación multiusuario, las tareas para garantizar consistencia ante
% una cancelación no son sencillas porque \ldots 

En este trabajo se propone una posible solución a estos problemas basada en
\emph{programación orientada a aspectos}.
Un primer aspecto llamado \emph{Observable} (POO) permite que, en forma
transparente, los objetos de dominio disparen eventos al ser modificados, de
forma que la interfaz de usuario pueda actualizarse para reflejar esos cambios.

Un segundo aspecto llamado \emph{Transaccional} (POT) automatiza la cancelación
de una operación garantizando que todos los objetos quedan en el mismo estado en
el que estaban antes de comenzar. Esto permite que las modificaciones a los
objetos del dominio se realicen respetando el concepto de transacción, en
particular las propiedades de atomicidad y consistencia.

Para desarrollar POT y POO se construyó una herramienta llamada
\emph{Aspects for Pure Objects} (APO).
APO es una herramienta para programacion orientada a aspectos que abstrae
conceptos fundamentales y complejos del framework \emph{Javassist}. 
APO es independiente de POT y POO, se puede utilizar
para la creación de otros aspectos.

%Si bien las herramientas desarrolladas son independientes entre sí, también se
% desarrollaron herramientas auxiliares que permiten integrarlas.
Para mostrar la posibilidad de componer los aspectos POO y POT se los integró en
una extensión del framework Arena.
Arena es un framework educativo para la creacion de interfaces de usuario
utilizado.
Las extensiones desarrolladas en el marco de este trabajo han sido incorporadas
a la última versión del Arena que ya está siendo utilizada en la materia de
Construcción de UIs de la Universidad Nacional de Quilmes, entre otras.

%Esta integracion tiene como objetivo facilitar el uso del Arena, ya que se
% utiliza con fines educativos.
  

\ldots TODO\ldots

		MVC
		Eventos
		Framework Arena
	Transacciones
		Propiedades ACID
	Programación orientada a Aspectos
Objetivo
Solucion propuesta
	Aspecto Observable
 	Aspecto Transaccional
 	Integración de ambos aspectos entre sí y con la UI 
Nuestra herramienta
 	Desarrollo de Aspect for Pure Objects
 	Pure Object Transaction
 	Pure Observable Objects
 	Integración de POT, POO y Arena
