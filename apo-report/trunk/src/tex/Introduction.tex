\section{Introducción}

Este trabajo se centra en automatizar dos tareas comunes en el desarrollo
de aplicaciones orientadas a objetos. El primero es el traspaso de datos
entre los objetos del dominio y los componentes de la interfaz gráfica. Y el
segundo es tener un mecanismo de cancelación de las operaciones.
En una aplicación multiusuario, las tareas para garantizar consistencia ante una cancelación no
son sencillas. En este trabajo se propone una posible solución a estos
problemas basada en programación orientada a aspectos. Un primer aspecto
permite que los objetos de dominio en forma transparente disparen eventos al
ser modificados, de forma de actualizar la interfaz de usuario (UI) acorde a
esos cambios. Un segundo aspecto automatiza las operatoria relacionada con las
cancelaciones, permitiendo que las modificaciones a los objetos del dominio se
realicen siguiendo el concepto de transacción, y concentrándonos en las
propiedades de atomicidad y consistencia.



\ldots TODO:Sin revisión\ldots

Contexto
	Construcción de interfaces de usuario usando el patrón MVC
		MVC
		Eventos
		Binding
		Framework Arena
	Transacciones
		Propiedades ACID
		Niveles de aislamiento
	Programación orientada a Aspectos
		Definición de un Aspecto
Estado del arte
	Estrategias para la comunicación dominio-vista
		Binding con eventos manuales
		Formularios
	Manejo de transacciones en la UI
Objetivo
Solucion propuesta
	Aspecto Observable
 	Aspecto Transaccional
 	Integración de ambos aspectos entre sí y con la UI 
Nuestra herramienta
	Selección de un framework de aspectos
 	Desarrollo de Aspect for Pure Objects
 	Pure Object Transaction
 	Pure Observable Objects
 	Integración de POT, POO y Arena
  	Otras mejoras al Arena 
Aplicacion d 	