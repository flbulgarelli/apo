\section{Metodología utilizada}
\label{sec:Methodology}

{\bf Programacion orientada a Aspectos }
		
Ya sea a través de la POO o con otras técnicas de abstracción de alto nivel, se
logra un diseño y una implementación que satisface la funcionalidad básica, y con
una calidad aceptable. Sin embargo, existen conceptos que no pueden encapsularse
dentro de una unidad funcional, debido a que atraviesan todo el sistema, o varias
parte de él (crosscutting concerns). Algunos de estos conceptos son: sincronización,
manejo de memoria, distribución, chequeo de errores, profiling, seguridad o redes,
entre otros. Así lo muestran los siguientes ejemplos:

\begin {enumerate}

	\item
	Manejo de errores y de fallas: agregar a un sistema simple un buen
	manejo de errores y de fallas requiere muchos y pequeños cambios y
	adiciones por todo el sistema debido a los diferentes contextos
	dinámicos que pueden llevar a una falla, y las diferentes políticas
	relacionadas con el manejo de una falla \cite{Kicz97a}.
	
	\item
	Consideremos una aplicación que incluya conceptos de seguridad y
	sincronización, como por ejemplo, asegurarnos que dos usuarios no
	intenten acceder al mismo dato al mismo tiempo. Ambos conceptos
	requieren que los programadores escriban la misma funcionalidad en
	varias partes de la aplicación. Los programadores se verán forzados a
	recordar todas estas partes, para que a la hora de efectuar un cambio
	y / o una actualización puedan hacerlo de manera uniforme a través
	de todo el sistema. Tan solo olvidarse de actualizar algunas de estas
	repeticiones lleva al código a acumular errores. \cite{Aquila}

\end{enumerate}

Para resolver estos problemas  recorremos a implementaciones oscuras, con
algunas falencias en el codigo, que en un futuro nos pueden trarer problemas.
Algunos síntomas de este problema pueden ser categorizados de la siguiente manera:

\begin {enumerate}

	\item {\bf Código Mezclado } \emph{(Code Tangling)}:
	En un mismo módulo de un sistema de software pueden simultáneamente convivir más
	de un requerimiento. Esta múltiple existencia de requerimientos lleva a la
	presencia conjunta de elementos de implementación de más de un
	requerimiento, resultando en un Código Mezclado.\cite{mislevy2008dpa}
	
	\item {\bf Código Diseminado} \emph{(Code Scattering)}:
	Como los requerimientos están esparcidos sobre varios módulos, la implementación
	resultante también queda diseminada sobre esos módulos.\cite{mislevy2008dpa}
 
\end{enumerate}

Estos síntomas combinados afectan tanto el diseño como el desarrollo de
software, de diversas maneras

\begin{itemize}
  
	\item Menor productividad: La implementación simultánea de múltiples
	conceptos distrae al desarrollador del concepto principal, por
	concentrarse también en los conceptos periféricos, disminuyendo la
	productividad.
	\cite{Fradet}
  
  \item Menor reuso: Al tener en un mismo módulo implementados varios
	conceptos, resulta en un código poco reusable.
	
	\item Baja calidad de código: El Código Mezclado produce un código
	propenso a errores. Además, al tener como objetivo demasiados
	conceptos al mismo tiempo se corre el riesgo de que algunos de ellos
	sean subestimados.\cite{Fradet}
	
	\item Evolución más dificultosa: Como la implementación no está
	completamente modularizada los futuros cambios en un
	requerimiento implican revisar y modificar cada uno de los módulos
  
\end{itemize}

Como respuesta a estos problemas nace la Programación Orientada a Aspectos
(POA). La POA permite a los programadores escribir, ver y editar un aspecto
diseminado por todo el sistema como una entidad por separado, de una manera
inteligente, eficiente e intuitiva.\\ \\

La POA es una metodología de programación que soporta la
separación de las propiedades para los aspectos antes mencionados. Esto implica
separar la funcionalidad básica y los aspectos, y los aspectos entre sí, a través de
mecanismos que permitan abstraerlos y componerlos para formar todo el sistema.
\\\\

La POA es un desarrollo que sigue a la POO, y como tal, soporta la
descomposición orientada a objetos, además de la procedimental y la funcional. Sin
embargo, la programación orientada a aspectos no es una extensión de la POO, ya
que puede utilizarse con los diferentes estilos de programación mencionados
anteriormente.







