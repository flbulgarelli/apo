\section{Metodología utilizada}
\label{sec:Methodology}
Introduccion a Aspectos
		1 en general
		en particular, xq aplica en este

{\bf Programacion orientada a Aspectos }
		
Ya sea a través de la POO o con otras técnicas de abstracción de alto nivel, se
logra un diseño y una implementación que satisface la funcionalidad básica, y con
una calidad aceptable. Sin embargo, existen conceptos que no pueden encapsularse
dentro de una unidad funcional, debido a que atraviesan todo el sistema, o varias
parte de él (crosscutting concerns). Algunos de estos conceptos son: sincronización,
manejo de memoria, distribución, chequeo de errores, profiling, seguridad o redes,
entre otros. Así lo muestran los siguientes ejemplos:

\begin {enumerate}

\item
Manejo de errores y de fallas: agregar a un sistema simple un buen
manejo de errores y de fallas requiere muchos y pequeños cambios y
adiciones por todo el sistema debido a los diferentes contextos
dinámicos que pueden llevar a una falla, y las diferentes políticas
relacionadas con el manejo de una falla \cite{Kicz97a}.

\item
Consideremos una aplicación que incluya conceptos de seguridad y
sincronización, como por ejemplo, asegurarnos que dos usuarios no
intenten acceder al mismo dato al mismo tiempo. Ambos conceptos
requieren que los programadores escriban la misma funcionalidad en
varias partes de la aplicación. Los programadores se verán forzados a
recordar todas estas partes, para que a la hora de efectuar un cambio
y / o una actualización puedan hacerlo de manera uniforme a través
de todo el sistema. Tan solo olvidarse de actualizar algunas de estas
repeticiones lleva al código a acumular errores. \cite{Aquila}


 